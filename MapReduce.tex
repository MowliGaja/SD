% Options for packages loaded elsewhere
\PassOptionsToPackage{unicode}{hyperref}
\PassOptionsToPackage{hyphens}{url}
%
\documentclass[
]{article}
\usepackage{lmodern}
\usepackage{amssymb,amsmath}
\usepackage{ifxetex,ifluatex}
\ifnum 0\ifxetex 1\fi\ifluatex 1\fi=0 % if pdftex
  \usepackage[T1]{fontenc}
  \usepackage[utf8]{inputenc}
  \usepackage{textcomp} % provide euro and other symbols
\else % if luatex or xetex
  \usepackage{unicode-math}
  \defaultfontfeatures{Scale=MatchLowercase}
  \defaultfontfeatures[\rmfamily]{Ligatures=TeX,Scale=1}
\fi
% Use upquote if available, for straight quotes in verbatim environments
\IfFileExists{upquote.sty}{\usepackage{upquote}}{}
\IfFileExists{microtype.sty}{% use microtype if available
  \usepackage[]{microtype}
  \UseMicrotypeSet[protrusion]{basicmath} % disable protrusion for tt fonts
}{}
\makeatletter
\@ifundefined{KOMAClassName}{% if non-KOMA class
  \IfFileExists{parskip.sty}{%
    \usepackage{parskip}
  }{% else
    \setlength{\parindent}{0pt}
    \setlength{\parskip}{6pt plus 2pt minus 1pt}}
}{% if KOMA class
  \KOMAoptions{parskip=half}}
\makeatother
\usepackage{xcolor}
\IfFileExists{xurl.sty}{\usepackage{xurl}}{} % add URL line breaks if available
\IfFileExists{bookmark.sty}{\usepackage{bookmark}}{\usepackage{hyperref}}
\hypersetup{
  pdftitle={MapReduce},
  pdfauthor={V Y Gajarajan},
  hidelinks,
  pdfcreator={LaTeX via pandoc}}
\urlstyle{same} % disable monospaced font for URLs
\usepackage[margin=1in]{geometry}
\usepackage{graphicx,grffile}
\makeatletter
\def\maxwidth{\ifdim\Gin@nat@width>\linewidth\linewidth\else\Gin@nat@width\fi}
\def\maxheight{\ifdim\Gin@nat@height>\textheight\textheight\else\Gin@nat@height\fi}
\makeatother
% Scale images if necessary, so that they will not overflow the page
% margins by default, and it is still possible to overwrite the defaults
% using explicit options in \includegraphics[width, height, ...]{}
\setkeys{Gin}{width=\maxwidth,height=\maxheight,keepaspectratio}
% Set default figure placement to htbp
\makeatletter
\def\fps@figure{htbp}
\makeatother
\setlength{\emergencystretch}{3em} % prevent overfull lines
\providecommand{\tightlist}{%
  \setlength{\itemsep}{0pt}\setlength{\parskip}{0pt}}
\setcounter{secnumdepth}{-\maxdimen} % remove section numbering

\title{MapReduce}
\author{V Y Gajarajan}
\date{01/09/2020}

\begin{document}
\maketitle

\hypertarget{overview}{%
\subsection{Overview}\label{overview}}

\begin{itemize}
\tightlist
\item
  Sequentially read a lot of data
\item
  \textbf{Map:}

  \begin{itemize}
  \tightlist
  \item
    Extract something you care about
  \end{itemize}
\item
  \textbf{Group by key:}Sort and Shuffle
\item
  \textbf{Reduce:}

  \begin{itemize}
  \tightlist
  \item
    Aggregate, summarize, filter or transform\\
    -Write the result\\
    \#\#\#\# Map Reduce Word Counting
    \includegraphics{C:/Users/gajar/Desktop/ctr/wordcount.png}
  \end{itemize}
\end{itemize}

\hypertarget{shallow-of-concept}{%
\subsection{Shallow of Concept}\label{shallow-of-concept}}

\hypertarget{morespecifically}{%
\subsubsection{MoreSpecifically}\label{morespecifically}}

\begin{itemize}
\tightlist
\item
  \textbf{Input:} a set of key-value pairs
\item
  Programmer specifies two methods:\\
  -\textbf{Map(k, v)} → \textless k', v'\textgreater* -Takes a key-value
  pair and outputs a set of key-value pairs -E.g., key is the filename,
  value is a single line in the file -There is one Map call for every
  (k,v) pair\\
  -\_\_Reduce(k', \textless v'\textgreater{}\emph{)\_\_ → \textless k',
  v''\textgreater{}} -\emph{All values v' with same key k' are reduced
  together and processed in v' order} -There is one Reduce function call
  per unique key k'
\end{itemize}

\hypertarget{mapreduce-the-map-step}{%
\paragraph{MapReduce :The Map Step}\label{mapreduce-the-map-step}}

\begin{figure}
\centering
\includegraphics{C:/Users/gajar/Desktop/ctr/mapstep.png}
\caption{The Map Step.}
\end{figure}

\hypertarget{mapreduce-the-reduce-step}{%
\paragraph{MapReduce :The Reduce Step}\label{mapreduce-the-reduce-step}}

\begin{figure}
\centering
\includegraphics{C:/Users/gajar/Desktop/ctr/reducestep.png}
\caption{The Reduce Step.}
\end{figure}

\hypertarget{deep-of-concept}{%
\subsection{Deep of Concept}\label{deep-of-concept}}

\hypertarget{word-count-using-mapreduce-pseudocode}{%
\subsubsection{Word Count Using MapReduce
Pseudocode}\label{word-count-using-mapreduce-pseudocode}}

\begin{verbatim}
map(key, value): 
// key: document name; value: text of the document
  for each word w in value:
    emit(w, 1) 
reduce(key, values):
// key: a word; value: an iterator over counts
  result = 0
  for each count v in values:
    result += v
  emit(key, result)
\end{verbatim}

\hypertarget{map-reduceenvironment}{%
\subsubsection{Map-Reduce:Environment}\label{map-reduceenvironment}}

\textbf{Map-Reduce environment takes care of:} - Partitioning the input
data\\
- Scheduling the program's execution across a set of machines\\
- Performing the group by key step\\
- Handling machine failures - Managing required inter-machine
communication

\hypertarget{mapreduce-a-diagram}{%
\paragraph{Mapreduce: A diagram}\label{mapreduce-a-diagram}}

\begin{figure}
\centering
\includegraphics{C:/Users/gajar/Desktop/ctr/mapreducedia.png}
\caption{Mapreduce: A diagram.}
\end{figure}

\hypertarget{mapreduce-in-parallel}{%
\paragraph{Mapreduce: In Parallel}\label{mapreduce-in-parallel}}

\begin{figure}
\centering
\includegraphics{C:/Users/gajar/Desktop/ctr/mapreducepar.png}
\caption{Mapreduce: In Parallel.}
\end{figure}

\end{document}
